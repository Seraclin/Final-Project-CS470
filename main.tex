\documentclass{article}

% Language setting
% Replace `english' with e.g. `spanish' to change the document language
\usepackage[english]{babel}

% Set page size and margins
% Replace `letterpaper' with `a4paper' for UK/EU standard size
\usepackage[letterpaper,top=2cm,bottom=2cm,left=3cm,right=3cm,marginparwidth=1.75cm]{geometry}


% Useful packages
\usepackage{amsmath}
\usepackage{graphicx}
\usepackage[colorlinks=true, allcolors=blue]{hyperref}
\usepackage{adjustbox}
\usepackage{url}
\usepackage{pdfpages} % for pdfs
\usepackage{nameref} % name refs


\title{CS 470 Final Project}
\author{Samantha Lin, Yuritzy Ramos, Matthew Joeseop, Ruilin Chen}

\begin{document}
\maketitle

\section{Collaboration Statement}
I did not collaborate with anyone, receive any assistance nor use any external sources not cited below.

\section{Problem}
%  Describe the problem that you are trying to solve in this project, and the goals you want to achieve
\textbf{Page Rank: Upgrade and Expand or Retire?}
\newline 
Our team hopes to classify the relative importance/popularity of different bike stations given in the data based on how frequently they appear as start and end destinations for different trips. There were two methods to approach this. First, we could construct a MultiDigraph from the data which would allow for multiple edges between nodes (i.e., five people took a trip from station 380 to 250). This would give a more realistic outcome about a station's usage based on user transactions/trips. However, including these edges may be redundant since they represent the same path/route and will result in the page rank scores of each station being lower. The second method is to use a DiGraph which ignores multiple edges between nodes. This means that it only adds an edge between nodes once, even if the edge appears multiple times in the data. This would result in a page rank outcome based on the stations’  interconnectivity rather than how many customers frequent a specific route. This would allow us to observe which stations have the most in-coming links/trips from other stations and how this impacts their importance in the network. We expect there to be a group of stations with higher levels of importance than others and hope to use this information to determine which stations should be expanded/upgraded to increase customer usage. This information could also be used to determine which stations are less frequented and should be removed from the network (i.e., retiring the station).
\section{Dataset}

\section{Preprocessing}

\section{Methods}
\textbf{Page Rank: NetworkX vs. MDD Implementation}
\newline
We used PageRank to rank the stations by level of importance in the network. Page Rank is frequently used to determine the importance of web pages based on the number and weight of incoming-links from other web pages. In our case, stations serve as web pages and trips serve as links between them. Stations with more incoming-links are expected to have a higher ranking; however, links from stations that have high rankings will boost the page rank score of the stations they point to.
\newline
We decided to use the page rank function from NetworkX since it can take both DiGraphs and MultiDiGraphs as inputs. By default, the page rank algorithm in NetworkX uses a dampening factor of 0.85 and a maximum of 1000 iterations. For comparison, we also used the page rank implementation from the Mining of Massive Datasets (MMD) textbook to obtain page rank scores for each station. Similar to the NetworkX page rank implementation, we used a dampening factor of 0.85  and a maximum of 1000 iterations.


\section{Results}
For the NetworkX implementation of page rank,  the start and end station IDs for each trip were obtained from the data set. These were stored in a tuple of form (start, end) and appended to a list of edges for later use. Once all trips were stored, the edges were added to either a NetworkX Digraph or MultiDigraph. The main difference between the two is that whole MultiDiGraph allows for multiple edges between nodes, DiGraph only counts the appearance of an edge once. The resulting graph was then passed to the NetworkX page rank function with dampening parameter of 0.85 and max iterations 1000. The page rank scores  were sorted in descending order and the station IDs were sorted in ascending order. The top 10 results for the graphs with and without duplicate edges can be found in the tables below: 
\begin{table}[h!]
\caption{Page Rank Results for Stations (Duplicate Edges)}
\centering
 \begin{tabular}{||c c||} 
 \hline
 Station & Page Rank \\ [0.5ex] 
 \hline\hline
 519 & 0.00629\\ 
 497 & 0.00563\\
 426 & 0.00503\\ 
 435 & 0.00489\\
 293 & 0.00475\\
 387 & 0.00468\\
 402 & 0.004619\\
 151 & 0.004617\\
 499 & 0.00430\\
 285 & 0.00425\\[1ex] 
 \hline
 \end{tabular}
 \end{table}

 \begin{table}[h!]
\caption{Page Rank Results for Stations (No Duplicate Edges)}
\centering
 \begin{tabular}{||c c||} 
 \hline
 Station & Page Rank \\ [0.5ex] 
 \hline\hline
 151 & 0.00298\\ 
 519 & 0.00291\\
 497 & 0.00289\\ 
 532 & 0.002762\\
 499 & 0.00269581\\
 387 & 0.002623\\
 251 & 0.002603\\
 402 & 0.0025953\\
 323 & 0.0025923\\
 293 & 0.002586\\[1ex] 
 \hline
 \end{tabular}
 \end{table}
For the MMD implementation of page rank, two separate dot files were created as input. One is a dot file with all 473,556 edges between the 840 nodes and the other contains no duplicate edges, reducing the number of edges to 107,649. The page rank scores (with dampening factor 0.85 and max iteration 1000) were sorted in descending order and the station IDs were sorted in ascending order. The top 10 results for the graphs with and without duplicate edges can be found in the tables below: 

\begin{table}[h!]
\caption{Page Rank Results for Stations (Duplicate Edges)}
\centering
 \begin{tabular}{||c c||} 
 \hline
 Station & Page Rank \\ [0.5ex] 
 \hline\hline
 151 & 0.002991\\ 
 519 & 0.002913\\
 497 & 0.00288951\\ 
 532 & 0.002761\\
 499 & 0.002670\\
 387 & 0.002626\\
 251 & 0.002610\\
 402 & 0.0025997\\
 323 & 0.002593\\
 293 & 0.002592\\[1ex] 
 \hline
 \end{tabular}
 \end{table}
 
 \begin{table}[h!]
\caption{Page Rank Results for Stations (No Duplicate Edges)}
\centering
 \begin{tabular}{||c c||} 
 \hline
 Station & Page Rank \\ [0.5ex] 
 \hline\hline
 151 & 0.002991\\ 
 519 & 0.002912\\
 497 & 0.0028985\\ 
 532 & 0.002761\\
 499 & 0.002698\\
 387 & 0.002625\\
 251 & 0.002610\\
 402 & 0.0025997\\
 323 & 0.002593\\
 293 & 0.002592\\[1ex] 
 \hline
 \end{tabular}
 \end{table}
 \textbf{Comparison of results:}
 \newline
	For the top ten results of the trips graphs with duplicate edges, there was a 70\% match in the stations between the two methods we implemented - stations 151, 519, 497, 499, 387, 402, and 293. The three dissimilarities that occurred were attributed to the random walk aspect of the two algorithms and the influence of the multiple edges on page rank. Although the ranking position for the stations were not always the same, we found it sufficient to approximate the top ten stations for our purposes and so the order does not affect our decisions as far as which stations are the most frequented by customers. It was interesting to note, however, that the bottom ten stations (the least frequented stations) had the same IDs and rankings for the two methods. These were stations  3468, 3017, 3240, 3014, 3450, 3485, 3506, 3557, 3607, and 3636.
 \newline
When it came to the graphs without duplicate edges, our results were a 100\% match in both rankings and station IDs. The top ten stations were 51, 519, 497, 532, 499, 387, 251, 402, 323, and 293. The bottom ten stations were the same as in the previous example, stations 3468, 3017, 3240, 3014, 3450, 3485, 3506, 3557, 3607, and 3636. The similarity between the two page rank methods used for this graph case reveals a potential downside to using the original data with duplicate edges as it produces less consistent rankings depending on the method used compared to singly linked graphs. 
\section{Future Works}


\end{document}
