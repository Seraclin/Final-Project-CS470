\documentclass{article}

% Language setting
% Replace `english' with e.g. `spanish' to change the document language
\usepackage[english]{babel}

% Set page size and margins
% Replace `letterpaper' with `a4paper' for UK/EU standard size
\usepackage[letterpaper,top=2cm,bottom=2cm,left=3cm,right=3cm,marginparwidth=1.75cm]{geometry}


% Useful packages
\usepackage{amsmath}
\usepackage{graphicx}
\usepackage[colorlinks=true, allcolors=blue]{hyperref}
\usepackage{adjustbox}
\usepackage{url}
\usepackage{pdfpages} % for pdfs
\usepackage{nameref} % name refs


\title{CS 470 Final Project}
\author{Samantha Lin, Yuritzy Ramos, Matthew Joeseop, Ruilin Chen}

\begin{document}
\maketitle

\section{Collaboration Statement}
I did not collaborate with anyone, receive any assistance nor use any external sources not cited below.

\section{Problem}
%  Describe the problem that you are trying to solve in this project, and the goals you want to achieve
\textbf{Page Rank: Upgrade and Expand or Retire?}
\newline 
Our team hopes to classify the relative importance/popularity of different bike stations given in the data based on how frequently they appear as start and end destinations for different trips. There were two methods to approach this. First, we could construct a MultiDigraph from the data which would allow for multiple edges between nodes (i.e., five people took a trip from station 380 to 250). This would give a more realistic outcome about a station's usage based on user transactions/trips. However, including these edges may be redundant since they represent the same path/route and will result in the page rank scores of each station being lower. The second method is to use a DiGraph which ignores multiple edges between nodes. This means that it only adds an edge between nodes once, even if the edge appears multiple times in the data. This would result in a page rank outcome based on the stations’  interconnectivity rather than how many customers frequent a specific route. This would allow us to observe which stations have the most in-coming links/trips from other stations and how this impacts their importance in the network. We expect there to be a group of stations with higher levels of importance than others and hope to use this information to determine which stations should be expanded/upgraded to increase customer usage. This information could also be used to determine which stations are less frequented and should be removed from the network (i.e., retiring the station).
\section{Dataset}

\section{Preprocessing}

\section{Methods}
\textbf{Page Rank: NetworkX vs. MDD Implementation}
\newline
We used PageRank to rank the stations by level of importance in the network. Page Rank is frequently used to determine the importance of web pages based on the number and weight of incoming-links from other web pages. In our case, stations serve as web pages and trips serve as links between them. Stations with more incoming-links are expected to have a higher ranking; however, links from stations that have high rankings will boost the page rank score of the stations they point to.
\newline
We decided to use the page rank function from NetworkX since it can take both DiGraphs and MultiDiGraphs as inputs. By default, the page rank algorithm in NetworkX uses a dampening factor of 0.85 and a maximum of 1000 iterations. For comparison, we also used the page rank implementation from the Mining of Massive Datasets (MMD) textbook to obtain page rank scores for each station. Similar to the NetworkX page rank implementation, we used a dampening factor of 0.85  and a maximum of 1000 iterations.


\section{Results}

\section{Future Works}


\end{document}
